% Paper Template - Cross-Platform Sentiment Analysis
% Use Overleaf or local LaTeX installation
% Template: IEEEtran for conference style

\documentclass[conference]{IEEEtran}
\usepackage{graphicx}
\usepackage{amsmath}
\usepackage{booktabs}
\usepackage{hyperref}
\usepackage{algorithm}
\usepackage{algorithmic}

\title{Cross-Platform Sentiment Analysis Using Context-Aware Meta-Learning: A Study of YouTube and News Media}

\author{
    \IEEEauthorblockN{[Your Name]}
    \IEEEauthorblockA{
        [Your Institution]\\
        [Your Email]
    }
}

\begin{document}

\maketitle

\begin{abstract}
Sentiment analysis across heterogeneous platforms presents unique challenges due to differing linguistic styles and content formats. We present a context-aware meta-learning approach that adaptively weights multiple sentiment analyzers based on text characteristics and source metadata. Our method combines VADER, RoBERTa, and DistilBERT in a Random Forest meta-classifier. Evaluated on 125 manually annotated YouTube videos across 25 controversial topics, our approach achieves [X]\% accuracy, significantly outperforming baselines (p < 0.05). We introduce a sentiment divergence metric for credibility assessment and release our dataset publicly.
\end{abstract}

\section{Introduction}
% TODO: Write introduction (1.5 pages)
% - Motivation
% - Problem statement  
% - Research questions
% - Contributions

\section{Related Work}
% TODO: Cite 30+ papers (2 pages)

\subsection{Sentiment Analysis Methods}
% VADER, BERT-based models, etc.

\subsection{Ensemble Approaches}
% Voting, stacking, meta-learning

\subsection{YouTube and Video Analysis}
% Prior work on video sentiment

\subsection{Cross-Platform Mining}
% Multi-source analysis

\section{Methodology}

\subsection{Dataset Construction}
% Annotation process, guidelines, inter-annotator agreement

\subsection{Base Sentiment Analyzers}
% VADER, RoBERTa, DistilBERT descriptions

\subsection{Context-Aware Meta-Learner}
% Feature extraction, Random Forest, training

\subsection{Cross-Platform Divergence Metric}
% Mathematical definition of SDS

\section{Experiments}

\subsection{Experimental Setup}
\subsection{Baseline Methods}
\subsection{Evaluation Metrics}

\section{Results}

\subsection{Sentiment Classification Performance}
% Table: Baseline comparison

\subsection{Feature Importance Analysis}
% Figure: Feature importance

\subsection{Divergence-Credibility Correlation}
% Statistical tests

\subsection{Error Analysis}

\section{Discussion}

\subsection{Why Meta-Learning Works}
\subsection{Limitations}

\section{Conclusion}

\section*{Data and Code Availability}
Dataset and code: \url{https://github.com/aryajayasankar/ExplainNet}

\bibliographystyle{IEEEtran}
\bibliography{references}

\end{document}
